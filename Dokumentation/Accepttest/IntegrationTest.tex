\documentclass[Main]{subfiles}

\begin{document}

\chapter{Integrationstests}

Når de enkelte enheder var testet blev større dele sammensat for at se, om de kunne spille sammen.
Ved at køre det store projekt kan forskellige integrationstests vises.

\subsection{Dronen}
I filen \code{InoHelp.cpp} ligger følgende funktion, der køres 5 gange i sekundet:

\begin{lstlisting}[caption=process2HzTask()'s integrationstest, style=Code-C, label=lst:ref]
void process2HzTask()
{
	//PrintSonarReport();
	//PrintChosenProgram();
	//PrintControllerOutput();
	//PrintWarnings();
}
\end{lstlisting}

\begin{itemize}
\item Ved udkommentering af \code{PrintSonarReport()} printes dronens aktuelle højde, om den er i 'free flight' (ikke fastlagt højde) eller 'Fixed flight' (fastlagt højde), hvilken højden den skal ramme (såfremt den er i 'Fixed flight'), samt venstre-, front- og højre sonarsensorens aktuelle målinger.

\item Ved udkommentering af \code{PrintChosenProgram()} udskrives det valgte program.

\item Ved udkommentering af \code{PrintControllerOutput()} udskrives de værdier der sendes til \code{Receiver}-klassen for x-, y- og z-aksen, throttle-værdien samt om den skal holde sin højde eller ej.

\item Ved udkommentering af \code{PrintWarnings()} advarsler for hver sonar, såfremt der er nogen.
\end{itemize}


\newpage
I filen \code{ControlFaker.cpp} ligger følgende funtion, der køres 50 gange i sekunded.


\begin{lstlisting}[caption=lst:PrintMotorOutput()'s integrationstest, style=Code-C, label=lst:PrintMotorOutput]
void PrintMotorOutput()
{
	if(!IsMotorKilled())
	{
		//printInLine("Motor output: ", MOTORMODE);
		//printInLine(motorCommand[0], MOTORMODE);
		//printInLine(", ", MOTORMODE);
		//printInLine(motorCommand[1], MOTORMODE);
		//printInLine(", ", MOTORMODE);
		//printInLine(motorCommand[2], MOTORMODE);
		//printInLine(", ", MOTORMODE);
		//printInLine(motorCommand[3], MOTORMODE);
		//printInLine(", ", MOTORMODE);
		//printInLine(_controllerInput[THROTTLE], MOTORMODE);
		//println(MOTORMODE);
	}
}
\end{lstlisting}

Såfremt disse linjer udkommenteres vil den værdi, der sendes til hver enkelt motor på dronen, samt den værdi de skulle holde, blive udskrevet.
Dette er bl.a. brugt til at generere den graf der er vist i design-dokumentet\cite{Design}  i afsnit 2.1.1.4.










\end{document}