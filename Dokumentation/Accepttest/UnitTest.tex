\documentclass[Main]{subfiles}

\begin{document}

\chapter{Unit test}
Når koden skulle skrives til enkelte enheder og klasser blev delen, som skulle udvikles, testet med nogle små stumper kode, der ville fortælle om det virkede efter hensigten.
Herunder følger nogle af de tests der blev foretaget på de enkelte dele, for at se efter sammenspil.

\section{Dronen}
Dronen har kommunikeret med nogle enheder.
Disse følge herunder.

\subsection{Sonarsensorer}
Input fra sensorerne var allerede skrevet for én sensor, men den så også kun på denne.
For at aktivere kommunikationen til de sidste 3 og tage gennemsnittet af de sidste to målinger kunne følgende gøres:

\begin{itemize}
\item Opret et nyt Arduino-projekt.
\item Placer filerne \textit{MaxSonarRangeFinder.h}, \textit{QueueList.h}, \textit{RangeFinder.h}, \textit{SensorsStatus.h}, \textit{MaxSonarRangeFinder.cpp}, \textit{QueueList.cpp}, \textit{SensrosStatus.cpp} og \textit{RangeFinder.cpp} ved siden af den oprettede \code{.ino}-fil.
\item Indsæt følgende i \code{.ino}-filen:
\end{itemize}


\begin{lstlisting}[caption=Udskrift af sensorgennemsnit, style=Code-C, label=lst:ref]
#include "SensorsStatus.h"
#include "MaxSonarRangeFinder.h"

void setup() 
{
	Serial.begin(115200);
	inititalizeRangeFinders();
}

void loop() {
  
	updateRangeFinders();
  
	Serial.print("Distance = {");
	Serial.print(RangerAverage[ALTITUDE_RANGE_FINDER_INDEX].average);
	Serial.print(" : ");
	Serial.print(RangerAverage[FRONT_RANGE_FINDER_INDEX].average);
	Serial.print(" : ");
	Serial.print(RangerAverage[RIGHT_RANGE_FINDER_INDEX].average);
	Serial.print(" : ");
	Serial.print(RangerAverage[LEFT_RANGE_FINDER_INDEX].average);
	Serial.println("}");
	delay(100);
}
\end{lstlisting}
\begin{itemize}
\item Tilkobbel dronen til computeren med USB-kabel og åben en COM-port med 115200 baudrate.
\item Dronen vil nu skrive "\code{Distance = {x.xx : y.yy : z.zz : t.tt}}"\xspace som vil være gennemsnittet for de sidste to målinger foretaget på hver sonar.
Den første vil være fra sensoren under dronen, her efter front-, højre- og venstre-sonar.
\end{itemize}


\subsection{Registrering af modtageren}
For at læse gennem \itoc-forbindelse skal der først være forbindelse.
Dette kan testes ved procedure i Kodeudsnit \ref{lst:scanner}.
Indsæt dette i et nyt projekt og kør programmet.

Kodeudsnittet vil scanne alle enheder, der er sat til \itoc og printe dem ud på hver sin linje.
Radiomodtageren svarer med "\code{I2C device found at address 0x90}".

\begin{lstlisting}[caption=Scan efter \itoc enheder, style=Code-C, label=lst:scanner]
#include <Wire.h>

void setup()
{
  Wire.begin();

  Serial.begin(115200);
  Serial.println("\nI2C Scanner");
}

void loop()
{
  byte error, address;
  int nDevices;

  Serial.println("Scanning...");

  nDevices = 0;
  for(address = 1; address < 255; address++ ) 
  {
    Wire.beginTransmission(address);
    error = Wire.endTransmission();

    if (error == 0)
    {
      Serial.print("I2C device found at address 0x");
      if (address<16) 
        Serial.print("0");
      Serial.print(address,HEX);
      Serial.println("  !");

      nDevices++;
    }
    else if (error==4) 
    {
      Serial.print("Unknow error at address 0x");
      if (address<16) 
        Serial.print("0");
      Serial.println(address,HEX);
    }    
  }
  if (nDevices == 0)
    Serial.println("No I2C devices found\n");
  else
    Serial.println("done\n");

  delay(5000);           // wait 5 seconds for next scan
}
\end{lstlisting}


\newpage
\subsection{Læsning fra modtageren}
Når forbindelsen vha. \itoc er opsat kan læsninger påbegyndes.
\begin{itemize}
\item Opret et nyt projekt
\item Indsæt Kodeudsnit \ref{lst:itocLEs}
\item Kør programmet
\item Lad herefter radiomodtageren sende data - eks. ved tryk på fjernbetjeningen.
\item COM-porten vil udskrive data 10 gange i sekundet med et punktum, og såfremt der er registreret data, efterfulgt af programvalget.
\end{itemize}


\begin{lstlisting}[caption=title, style=Code-C, label=lst:itocLEs]
#include <Arduino.h>
#include <Wire.h>

#define TRANCEIVER_ADDRESS 0x90
#define INDICATORVALUE 0x3F

void SetupRadioCommunicaiton();
void ReadRadio();

int radioProgram;

void setup()
{
	Serial.begin(115200);            // start serial for output
	Serial.println("All setup");

	SetupRadioCommunicaiton();
}

void loop()
{
	delay(100);
	Serial.print(".");
	
	ReadRadio();
}

void SetupRadioCommunicaiton()
{
	Wire.begin(TRANCEIVER_ADDRESS);// join i2c bus with address 0x90
	radioProgram = 0;
}

void ReadRadio()
{
	int indicator, programValue;

	Wire.requestFrom(TRANCEIVER_ADDRESS, 2); //Request 2 bytes from radio

	indicator = Wire.read(); //First value is always an indicator if program is with or not
	_queue->push((float)Wire.read());

	programValue = _queue->PeekLastElementFilter(0);
	if(_queue->count() > 10)
		_queue->pop();

	if(indicator == INDICATORVALUE)
	{
		printInLine("New program: ", RADIOMODE);
		printNewLine(programValue, RADIOMODE);
	}
	radioProgram = programValue; //TODO: Valider dette
}
\end{lstlisting}

\newpage
\section{Kommunikation mellem µ-kontroller og radiomodul}
Test om disse to enheder kunne kommunikere.


\textbf{Opsætning:}\\
\textbf{Udstyr:}
\begin{itemize}
\item STK500 kit
\item Radio fra projektet
\item Adapter til stk500 kit
\end{itemize}


\textbf{Forbindelser:}
\begin{itemize}
\item PORTC -> Led port
\end{itemize}

\begin{itemize}
\item \textbf{Radio -> 	STK500}
\item CSN	->	PB2
\item GND	->	GND
\item SO	->	PB4
\item SCK	->	PB5
\item Si	->	PB3
\item +3.3V	->	VTG
\end{itemize}


Vtarget på STK500 skal sættes til 3.3V ned fra de normale 5V.

Radioen har nogle test register der kan læses fra. "Test0", "Test1" og "Test2" 

I dette forsøg læsers der fra Test2.
"This register will be forced to 0x88 or 0x81 when it wakes up from
SLEEP mode, depending on the configuration of FIFOTHR." \cite{TI-cc1101}

Med testkoden, skal svaret være "0x88". Resultatet kan ses på led porten på STK500 kittet.

\begin{lstlisting}[caption=SPI test, style=Code-C, label=lst:itocLEs]
// includes
#include <avr/io.h>
#include <util/delay.h>
#include <avr/sleep.h>


// variables

void SPI_MasterInit(void)         //Initialize Atmega8 as master
{
	DDRB = (1<<DDB5)|(1<<DDB3)|(1<<DDB2);   // Set MOSI,SCK and SS\ output, all others input
	SPCR = (1<<SPE)|(1<<MSTR)|(1<<SPR0);   //  Enable SPI, Master, set clock rate fck/16
}

//Toggle CSn
void CSn(int i)
{
	if(i==1)
	PORTB|=0x04;
	else
	PORTB&=0xFB;
}


unsigned char SPI_transmit(unsigned char data)
{
	// Start transmission
	SPDR = data;

	// Wait for transmission complete
	while(!(SPSR & (1<<SPIF)));
	data = SPDR;

	return(data);
}

int main()
{
    DDRC = 0xFF;
	SPI_MasterInit();         //initialize SPI
	
	CSn(0);
	//This register will be forced to 0x88 or 0x81 when it wakes up from SLEEP mode
	SPI_transmit(0xAE); //Read test 2 register
	PORTC = SPI_transmit(0x3B); //strobe - flush TX
	CSn(1);

	while (1)
	{

	}
}
\end{lstlisting}





\end{document}