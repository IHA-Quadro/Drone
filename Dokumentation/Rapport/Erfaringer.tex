\documentclass[Main]{subfiles}

\begin{document}


\subsection{Opnåede erfaringer}

Igennem projektet er der opnået en del erfaringer inden for nye felter:
\vspace{-20pt}
\begin{itemize}
\item Arduino
\vspace{-10pt}
\item[] Arduino er helt nyt for gruppens medlemmer og det har været både en udfordring, men også en fornøjelse at opleve nye sider af C og C++.
Med Arduino følger en masse biblioteker, som gør nogle aspekter meget nemt at gå til, mens Arduino's begrænsede mængde af biblioteker gør det svært at implementere specifikke design patterns.

\item Overtagelse af andres projekter
\vspace{-10pt}
\item[] Kodebasen for dronen er hentet fra Arduino's Github repository og dét, at sætte sig ind i andres kode, er en kæmpe udfordring, da den ikke ligner ens egen.


\item Sammenspil mellem hardware og software på større systemer
\vspace{-10pt}
\item[] At skrive software, som skal håndtere meget hardware i én tråd, er en stor udfordring som i starten kan se meget håbløs ud, før det store overblik viser sig.
Dette gør også, at alle funktioner skal skrives på en speciel måde.

\item Antenner
\vspace{-10pt}
\item[] En god antenne har stor betydning for et godt signal. 
Der findes mange forskellige designs, der hver især har sine egne fordele.

\item Radiosignaler
\vspace{-10pt}
\item[] Radiosignaler er komplekse, da frekvensens betydning for hastigheden, datamængde og rækkevidde er vigtige.

\item Kommunikationsprotokoller
\vspace{-10pt}
\item[] Hvor komplekst det kan være at implementere en sådan protokol. Det er vigtigt at overholde timing for, at få signaler igennem. 

\end{itemize}


\end{document}