\documentclass[Main]{subfiles}

\begin{document}


\subsection{Opnåede erfaringer}

Igennem projektet er der opnået en del erfaringer inden for nye felter:
\vspace{-20pt}
\begin{itemize}
\item Arduino
\item[] Arduino er helt nyt for gruppens medlemmer og det har været både en udfordring, men også en fornøjelse at opleve nye sider af C og C++.
Med Arduino følger en masse biblioteker, som gør nogle aspekter meget nemt at tilgå, mens Arduino's begrænsede mængde af biblioteker gør det svært at implementere specifikke design patterns.

\item Overtagelse af andres projekter
\item[] Kodebasen for dronen er taget fra Ardunio's Github repository og dét, at sætte sig ind i andres kode og stilarter, er en kæmpe udfordring, når det ikke ligner ens egen.


\item Sammenspil mellem hardware og software på større systemer
\item[] At skrive software, som skal håndtere meget hardware i én tråd, er en stor udfordring som i starten kan se meget håbløs ud, før det store overblik viser sig.
Dette gør også, at alle funktioner skal skrives på en speciel måde.

\item Antenner
\item[] Hvor stor betydning en god antenne har for at godt signal kan overraske én. Der findes mange forskellige designs der hver især har sine egne fordele.

\item Radiosignaler
\item[] Radiosignal er komplekse, da frekvensens betydning for hastigheden, datamængde og rækkevidde er vigtige.

\item Kommunikationsprotokoller
\item[] Hvor komplekst det kan være at implementere disse. Hvor vigtigt timings er at overholde for at få signaler igennem. 

\end{itemize}


\end{document}