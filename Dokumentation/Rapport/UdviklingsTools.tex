\documentclass[Main]{subfiles}

\begin{document}


\section{Udviklingsværktøjer}
Følgende afsnit beskriver de udviklingsværktøjer, der er brugt i løbet af projektet.

\subsection{Arduino}
Dronens software er skrevet til et Arduino-board, primært i sproget C og en lille del i C++.

\subsection{Microsoft Visual Studio 2012 Ultimate med Visual Micro plugin}
Selvom Arduino har sin egen IDE, er der brugt Visual Studio til redigering og kompilering af koden, da denne IDE giver et væsentligt bedre overblik, samt filsøgning, definitionssøgning og farvekoder.

\subsection{Atmel Studio 6}
Senderens og modtagerens software er skrevet i Atmel Studio, der har været den foretrukne IDE for embedded programmering.

\subsection{LucidChart}
Hjemmesiden LucidChart.com er brugt til at lave figurer, sekvensdiagrammer og UML-diagrammer af systemet.

\subsection{Github}
Til at gemme filer for projektet er Github.com blevet anvendt, således at gamle versioner af projektets kode og dokumenter kunne hentes frem igen.

\subsection{Texmaker}
Til at skrive dokumenter er Texmaker blevet anvendt.
Texmaker kompiler Latex-tekst til .pdf-filer og sikrer, sammen med Github, at flere kan arbejde i samme dokument uden store hindringer, når dokumenterne skal flettes.
Med Latex sikres et dokumentets layout, således der ikke opstår forskellige formateringer i de enkelte afsnit.

\subsection{Adobe Photoshop CS6}
Da nogle figurer har skraveringer af forskellig art, som krævede avanceret behandling, har Photoshop været benyttet hertil.

\subsection{RF Studio}
RF Studio er brugt til opsætning af radioerne. Programmet genererer registeropsætningen ud fra de valgte parameter.


\end{document}