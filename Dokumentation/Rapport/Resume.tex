\documentclass[Main]{subfiles}

\begin{document}

\chapter*{Resume}
\addcontentsline{toc}{chapter}{Resume}

Dette bachelorprojekt er udviklet for Ingeniørhøjskolen Aarhus Universitet og har til formål at udvikle en fjernbetjening til en drone af typen Cyclone fra firmaet AeroQuad, samt tilføje programmer der forenkler styringen af denne til en sikker styring.

Projektet har haft stor fokus på, at det skal kunne videreudvikles af andre studerende, således, at ny hardware vil være nemt at tilføje implementationsmæssigt.
Derudover har enkeltheden for brugeren af systemet haft fokus, da dette produkt skulle kunne betjenes af mange.

Selve udviklingen af produktet involverer valg og indkøb af drone og sender, videreudvikling af drone, design og implementering af fjernbetjening, samt kommunikation mellem dronen og fjernbetjeningen.
Dette er gjort over iterationer, således at gruppen har kunne overholde deadlines for udviklingen, samt gruppens vejleder vidste om gruppens tidsplan ville holde.

Projektet er velfungerende og nemt for andre studerende at overtage.
Brugeren af systemet kan lette dronen, lande, rotere den til siderne, flyve den op og ned, samt fremad.
Dronen er, så længe den er i luften, opmærksom på om der er forhindringer foran den og vil  selv flyve uden om disse.
Af sikkerhedsmæssige årsager kan den også stoppes på et vilkårligt tidspunkt og ved beskadigelse af propellerne vil den ligeledes stoppe automatisk.




















\chapter*{Abstract}
\addcontentsline{toc}{chapter}{Abstract}





















\end{document}