\documentclass[Main]{subfiles}

\begin{document}

\chapter*{Resume}
\addcontentsline{toc}{chapter}{Resume}

Dette bachelorprojekt er udviklet for Ingeniørhøjskolen Aarhus Universitet og har til formål at udvikle en fjernbetjening til en drone af typen Cyclone fra firmaet AeroQuad, samt tilføje styringsalgoritmer der forenkler styringen af denne til sikker styring.

Projektet har haft stor fokus på, at det skal kunne videreudvikles af andre studerende således, at nyt hardware vil være nemt at tilføje implementationsmæssigt.
Derudover har fokus lagt på, at brugeren nemt kan håndtere systemet.

Selve udviklingen af produktet involverer valg og indkøb af drone og sender, videreudvikling af dronen, design og implementering af fjern\-betjening, samt kommunikation mellem dronen og fjernbetjeningen.
Dette er gjort over iterationer, således at gruppen har kunne overholde deadlines for udviklingen, samt gruppens vejleder vidste om gruppens tidsplan ville holde.

Projektet er velfungerende og nemt for andre studerende at overtage.
Brugeren af systemet kan lette dronen, lande, rotere den til siderne, flyve den op og ned, samt fremad.
Dronen er, så længe den er i luften, opmærksom på om der er forhindringer foran den og vil  selv flyve uden om disse.
Af sikkerhedsmæssige årsager kan den også stoppes på et vilkårligt tidspunkt og ved beskadigelse af propellerne vil den ligeledes stoppe automatisk.




\chapter*{Abstract}
\addcontentsline{toc}{chapter}{Abstract}

This bachelor thesis is developed for Aarhus University School of Engineering and aims to develop a remote control for a drone of the type Cyclone manufactured by AeroQuad, as well as add control algorithms that simplify the management of this for safe control.

The project has been focusing on the option for other students to further develope so that new hardware will be easy to add further implementation.
In addition, the focus is on the user to easily handle the system.

The development of the product involves the selection and purchase of a drone and transmitter,  further development of the drone, design and implementation of the remote control and communication between the drone and remote control.
This is done in iterations so that the group could meet deadlines for development and the group's supervisor knew about the group's schedule would stick.

The project is functional and easy for other students to take over.
The user of the system can make the drone take off, land, rotate it to the sides, fly up and down and forward.
The drone will, as long as it is in the air, scan for any obstacles in front of it and will fly around these.
For safety reasons, it can also be stopped at any time and if damage to the propellers occurs will also stop the drone automatically.




















\end{document}