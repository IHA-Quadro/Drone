\documentclass[Main]{subfiles}

\begin{document}


\section{Projektgennemførelse}
Arbejdet for dette projekt er så godt som udelukkende foregået i lokale 331E, samt et tomt lokale i kælderen til prøveflyvning.
I dette afsnit beskrives tidsplanen for projektet, hvem der har haft ansvar for de enkelte dele af projektet, samt udviklingsmetoder brugt.

\subsection{Ansvarsområder}
I Tabel \ref{Tab:ansvar} er ansvaret for de enkelte områder angivet.
Rasmus Berg Kloster (RBK) og Rasmus Bækgaard (RAB).
\begin{table}[H]
\centering
	\begin{tabular}{l l}
	\hline
	\textbf{Ansvarsområde} & \textbf{Personinitialer} \\ \hline
	Design af fjernbetjening & RBK\\
	Udvikling af software til fjernbetjening & RBK\\
	Udvikling af software til dronen & RAB\\
	Kommunikation mellem drone og modtager & RBK, RAB\\
	Sonarkommunikation & RAB\\
	Rapport & RBK, RAB\\
	Kravspecifikation & RBK, RAB\\
	Foranalyse & RBK, RAB\\
	Accepttest & RBK, RAB
	\\ \hline
	\end{tabular}
\caption{Ansvarsområder}
\label{Tab:ansvar}
\end{table}


\subsection{Iterationer}
Projektet er bygget over iterationer og interne deadlines, der ville søger for, at projektet kunne blive afleveret til tiden:


\begin{enumerate}
\item[] Iteration 1
	\vspace{-10pt}
	\begin{itemize}
	\item Samle drone.
	\item Danne overblik over sender.
	\item Skrive omdrejningshastighed direkte til motorer.
	\end{itemize}
	
\item[] Iteration 2
	\vspace{-10pt}
	\begin{itemize}
	\item Etablere forbindelse mellem sender og modtager.
	\item Få styr på eksisterende kode og opbygning.
	\end{itemize}
	
\item[] Iteration 3
	\vspace{-10pt}
	\begin{itemize}
	\item Etablere usikker kommunikation mellem sender og modtager.
	\item Overskrive input til fiktiv sender på dronen.
	\item Skrive kommandoer via USB-kabel.
	\end{itemize}

\item[] Iteration 4
	\vspace{-10pt}
	\begin{itemize}
	\item Lave print af modtager.
	\item Implementere sonar sensorer på dronen.
	\item Påbegynde stabil flyvning.
	\end{itemize}
	
\item[] Iteration 5
	\vspace{-10pt}
	\begin{itemize}
	\item Etablere sikker forbindelse mellem sender og modtager med CRC.
	\item Stabil flyvning, samt rotation og fremad flyvning.
	\end{itemize}
	
\item[] Iteration 6
	\vspace{-10pt}
	\begin{itemize}
	\item Endeligt print af fjernbetjening.
	\item Test af drone med fjernbetjening.
	\end{itemize}

\end{enumerate}

\subsection{Tidsplan}
Projektet blev startet d. 26. august 2013 og blev afleveret d. 18. december.
Forløbet af de forskellige iterationer kan ses på Tabel \ref{Tab:Tidsplan}.

\begin{table}[H]
\centering
	\begin{tabular}{l r r}
	\hline
	\textbf{Iteration} & \textbf{Planlagt dato} & \textbf{Faktisk dato} \\ \hline
	Iteration 1 & 30/08 -13 & 30/08 -13\\
	Iteration 2 & 13/09 -13 & 17/09 -13\\
	Iteration 3 & 27/09 -13 & 26/09 -13\\
	Iteration 4 & 20/10 -13 & 16/10 -13\\
	Iteration 5 & 08/11 -13 & 12/11 -13\\
	Iteration 6 & 15/11 -13 & 15/11 -13\\
	Rapportskrivning & 13/12 -13 & 13/12 -13 \\
	Accepttest	& 16/12 -13 & 16/12 -13\\
	Afleveringsdato & 18/12 -13 & 
	\\ \hline
	\end{tabular}
\caption{Planlagte iterationer og deres faktiske datoer}
\label{Tab:Tidsplan}
\end{table}


\subsection{Projektstyring}
Projektet er lavet under gruppens vejleder, Torben Gregersen.
Der er blevet arbejdet alle ugens hverdage siden projektets start, fra 8:15 -- 16:00 og kun enkelte dage endnu længere.
Både RBK og RAB har haft ét fag i første kvartal, men kun RBK har haft i andet kvartal.
Hver morgen, såfremt begge deltagere af projektet var til stede, blev der holdt et kort statusmøde på, hvor langt de enkelte medlemmer i gruppen var med sine opgaver.















\end{document}
