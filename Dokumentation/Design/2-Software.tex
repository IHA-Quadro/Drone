\documentclass[Main]{subfiles}


\begin{document}

\chapter{Software}
Dette kapitel beskriver softwarens opbygning efter 4+1 modellen. \fxnote{Reference til GOF}

\section{Klassebeskrivelser}

\class{Drone.ino}
\code{Drone.ino} er hovedfilen for systemet og har ikke fået foretaget de store ændringer i forhold til den oprindelige kode.
Denne fil skal indeholde de to obligatoriske funktioner \code{void setup()} og \code {void loop()}, da dette er hvad Arduino kører som den første og anden funktion.
Funktionen \code{void loop()} er i grove træk tilsvarende til et \code{while(1)}-loop, der ikke kan stoppes.
\\
Først nulstilles de fleste af klasserne herfra, hvorefter et schedulerings-system startes med en timer, der ser på 100Hz-, 50Hz-, 3 forskellige 10Hz-, 2Hz og 1Hz-funktioner skal køres. 
Disse er vigtige for systemets opbygning, da dette tillader udskrift via serial-forbindelsen til debugging og fordi den kun har én tråd til håndtering af funktioner.

\begin{Function}
\name{initPlatform}
\para{void}
\retu{void}
\comm{Denne funktion mapper LED'er som in-/output og starter \itoc forbindelsen.}
\end{Function}


\begin{Function}
\name{measureCriticalSensors}
\para{void}
\retu{void}
\comm{Denne funktion gør kald til de vigtigste sensorer (gyrometer og accelero\-meter)}
\end{Function}


\begin{Function}
\name{Setup}
\para{void}
\retu{void}
\comm{Arduino's første obligatoriske funktion -- Starter og reset'er de fleste filer og klasser: \\
Baudrate for serial port, læsning fra EEPROM, kald til \code{initPlatform}, initierer motorer, ControlFaker, Receiver, Gyro (samt kalibrering), accelerometer, PID-værdier, kinematics, barometeret, sonarsensorerne samt nogle initieringsværdier for funktionen \code{loop()} og receiver'en. }
\end{Function}



\begin{Function}
\name{loop}
\para{void}
\retu{void}
\comm{Arduinno's anden obligatoriske funktion: Denne svarer til en main-funktion med et \code{while(1){}} loop. 
\\ 
For hvert loop måles den nuværende tid som timestamp og sammenlignes med den sidste, så den ved hvor lang tid der er gået.
Så kaldes \code{measureCriticalSensors}, da disse er kritiske for dronens stabilitet.
\\
Herefter inkrementeres \textit{FrameCounter}, der afgør hvornår fremtidige funktioner skal køres.
\textit{FrameCounter} indikerer, om der skal køres en 100-, 50-, tre forskellige 10-, 5-, eller 1-Hz funktion.
I slutningen af loopet sættes det sidste timestamp som den forrige tid, og hvis der er brug for det nulstilles \textit{FrameCounter}.}
\end{Function}


\class{inoHelper}
\code{inoHelper} er en hjælpefil der blev lavet til \code{drone.ino} på grund af dens uoverskuelige struktur og er den reelle implementering for 100Hz-, 50Hz-, tre forskellige 10Hz-, 5Hz-, og 1Hz-funktionerne.
Ud over den oprindelige implementering fra \code{drone.ino} er her også implementeret en masse debug-funktioner, som bruges til at måle alt fra sonarer input og motorer output til de falske værdier der sender til receiveren for at styre dronen efter ønske.


\begin{Function}
\name{initializePlatformSpecificAccelCalibration}
\para{void}
\retu{void}
\comm{Denne funktion indeholder nogle dummy-værdier for accelerometret, men disse overskrives af EEPROM's gemte værdier fra den reelle kalibrering.}
\end{Function}


\begin{Function}
\name{process100HzTask}
\para{void}
\retu{void}
\comm{Denne funktion beregner flere af de vigtige værdier for dronens stabilitet, herunder kald til gyroskopet, accelerometret, barometret og beregning af dronens hastighed.
\\
Derudover kaldes \code{FlightControlprocessor}'s funktion \code{ProcessFlightControl()}.}
\end{Function}



\begin{Function}
\name{process50HzTask}
\para{void}
\retu{void}
\comm{Denne funktion starter læsningen af kommandoer der skal udføres og opdaterer sonarsensorerne.}
\end{Function}


\begin{Function}
\name{process10HzTask1}
\para{void}
\retu{void}
\comm{Denne funktion er blevet overflødig, da den kun gør noget med en kompassensor på.}
\end{Function}



\begin{Function}
\name{process10HzTask2}
\para{void}
\retu{void}
\comm{Denne funktion læser serialporten for kommandoer, der bruges til debugging, samt sender svar retur fra dronen.}
\end{Function}



\begin{Function}
\name{process10HzTask3}
\para{void}
\retu{void}
\comm{Denne funktion laver en timestamp-opdatering for kinematic-funktionerne.}
\end{Function}



\begin{Function}
\name{process2HzTask}
\para{void}
\retu{void}
\comm{Denne funktion udskriver forskellige debug-information 2 gange i sekundet, herunder sonarsensorernes målinger, om disse er advarsler, det valgte program og styringsinput.}
\end{Function}



\begin{Function}
\name{process1HzTask}
\para{void}
\retu{void}
\comm{Denne funktion udskriver forskellige debug-information 1 gange i sekundet.}
\end{Function}



\begin{Function}
\name{PrintChosenProgram}
\para{void}
\retu{static void}
\comm{Denn funktion udskriver det valgte program.}
\end{Function}


\begin{Function}
\name{PrintSonarReport}
\para{void}
\retu{static void}
\comm{Denne funktion udskriver gennemsnitsværdien for de sidste målinger af sonarsensorerne og om dronen flyver i fri højde eller i en fixeret højde.}
\end{Function}



\begin{Function}
\name{PrintAltitudeReport}
\para{void}
\retu{static void}
\comm{Denne funktion udskriver, om dronen kan se sonarsensoren på maven og om den er i panic-mode.}
\end{Function}


\begin{Function}
\name{PrintDebugReport}
\para{void}
\retu{static void}
\comm{Denne funktion udskriver om motorerne er aktiveret.}
\end{Function}




\begin{Function}
\name{PrintWarnings}
\para{void }
\retu{static void}
\comm{Denne funktion udskriver sonarsensorernes advarsler, såfremt der er nogle.}
\end{Function}




\begin{Function}
\name{PrintControllerOutput}
\para{void}
\retu{static void}
\comm{Denne funktion udskriver receiver-input i form af x-, y- og z-aksen, gaspedalen og om den skal holde en bestemt højde.}
\end{Function}


















\end{document}
