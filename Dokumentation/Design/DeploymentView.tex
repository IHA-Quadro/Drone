\documentclass[Main]{subfiles}
\begin{document}

\section{Deployment View}
Systemet er bygget op af en fjernbetjening, dronen, samt modtagerenheden på dronen, vist på Figur \ref{Fig:DeploymentViewOverview}.

\begin{figure}[H]
\centering
\includegraphics[width = 0.90 \textwidth]{DeploymentViewOverview}
\caption{Deployment View diagram}
\label{Fig:DeploymentViewOverview}
\end{figure}

Fjernbetjeningen sender en frame til dronen vha. det trådløse 433 MHz-signal, som opfanges af modtager-enheden.
Herefter bliver pakken lagt over i en output buffer til dronen som. Dronen kan så hente denne pakke når den har tid til det. Kommunikationen mellem dronen og modtagern er gennem \itoc protokollen.


\newpage
\subsection{Protokoller}
En kort beskrivelse af protokollerne mellem de forskellige enheder.

\subsubsection{Fjernbetjening til modtager}
Ved tryk på knapperne på fjernbetjeningen sendes en frame til modtageren.
Signalet består af 2 bytes, hvoraf det første er en programindikator og den anden er selve programmet.

Kommandoer der kan sendes vises i Tabel \ref{Tab:kommandoer}.

\begin{table}[H]
  \centering
	\begin{tabular}{l l}
	\hline
	\textbf{Kommando} 	& \textbf{Indhold i bytes} \\ \hline
	Let 				& \code{0x03} + \code{0x3F} + \code{0x02} \\
	Autoland 			& \code{0x03} + \code{0x3F} + \code{0x04} \\
	Fremad 				& \code{0x03} + \code{0x3F} + \code{0x08} \\
	Roter til højre 	& \code{0x03} + \code{0x3F} + \code{0x0A} \\
	Roter til venstre 	& \code{0x03} + \code{0x3F} + \code{0x0C} \\
	Roter til højre 	& \code{0x03} + \code{0x3F} + \code{0x0E} \\
	Stop 				& \code{0x03} + \code{0x3F} + \code{0x10} \\
	Inkrementer højde 	& \code{0x03} + \code{0x3F} + \code{0x12} \\
	Dekrementer højde 	& \code{0x03} + \code{0x3F} + \code{0x14} \\ \hline	
  	\end{tabular}  
\caption{Kommandoer sendt sender og modtager}
\label{Tab:kommandoer}
\end{table}

\newpage
\subsubsection{Modtager til drone}
Når en frame er modtaget på modtagerradioen, tjekker den framen for om CRC passer. 
Hvis der er uoverensstemmelse, melle størrelsen på frame og CRC verdien, bliver framen kasseret. 
Hvis framens størrelse og CRC stemmer overens, laver modtagerradioen et interrupt på µ-controller 2. 
µ-Controlleren vil så hente den modtaget pakke fra radioen, strippe den for pakke længde. 
Herefter vil den blive lagt ud i en output buffer til \itoc forbindelsen til dronen. 
Som så kan hente den næste gang den har tid. 

\begin{table}[H]
  \centering
	\begin{tabular}{l l}
	\hline
	\textbf{Kommando} 	& \textbf{Indhold i bytes} \\ \hline
	Let 				& \code{0x3F} + \code{0x02} \\
	Autoland 			& \code{0x3F} + \code{0x04} \\
	Fremad 				& \code{0x3F} + \code{0x08} \\
	Roter til højre 	& \code{0x3F} + \code{0x0A} \\
	Roter til venstre 	& \code{0x3F} + \code{0x0C} \\
	Roter til højre 	& \code{0x3F} + \code{0x0E} \\
	Stop 				& \code{0x3F} + \code{0x10} \\
	Inkrementer højde 	& \code{0x3F} + \code{0x12} \\
	Dekrementer højde 	& \code{0x3F} + \code{0x14} \\ \hline	
  	\end{tabular}  
\caption{Kommandoer sendt mellem modtager og drone}
\label{Tab:kommandoer2}
\end{table}


\end{document}