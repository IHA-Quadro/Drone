\documentclass[Main]{subfiles}
\begin{document}

\section{Implementation View}


\subsection{Opsætning af udviklingsmiljø}

\subsubsection{Drone}
For at arbejde med dronen, skal der kompileres kode til Arduino-boardet, der sidder på dronen.
Til dette bruges en standardinstallation af programmet Visual Studio Ultimate 2012 eller Visual Studio Pro 2013 fra Microsoft, hvilket kan hentes fra Dreamspark's hjemmeside\footnote{\url{https://www.dreamspark.com/Student/Default.aspx}}.

Derudover skal selve koden til Arduino hentes fra Arduinos hjemmeside\footnote{\url{http://www.arduino.cc/en/Main/Software}}.
Arduino har sin egen IDE, som måske er meget god til små programmer, men da der til dronen bruges adskillige filer, begynder Arudino's IDE at miste sin styrke.
Koden kan derfor åbnes i Visual Studio.

For at kompilere koden til den rigtige processor skal Visual Studio dog vide, hvilke filer der skal bruges, da disse ikke er en del af standardbibliotekerne.
Disse bliver linket sammen med programmet Visual Micro\footnote{\url{http://www.visualmicro.com/page/Arduino-Visual-Studio-Downloads.aspx}}.
Opsætningen af Visual Micro kan ligeledes findes på Visual Micro's hjemmeside\footnote{\url{http://www.visualmicro.com/post/2011/10/04/How-to-test-a-new-installation-of- Arduino-for-Visual-Studio.aspx}}.

Projektet kan findes på Github\footnote{\url{https://github.com/IHA-Quadro/Drone}}, hvor alle tidligere version af projektet også kan findes.
Såfremt der ønskes at bruge Github til videreudvikling kan det gøres på to måder:

\begin{enumerate}
\item Opsæt Windows PowerShell til Github\footnote{\url{http://bakgaard.net/archives/19}} (anbefales for alle IKT'ere og elever med forståelse for computer)
	\begin{itemize}
	\item Tilføj hver bruger af projektet på projektets repository\footnote{\url{https://github.com/organizations/IHA-Quadro/teams/412487}}.
	\item Åben PowerShell og skriv din kode.
	\item Find ønsket placering for projektet (\code{cd Dokumenter/}).
	\item Klon repositoriet til en ny mappe kaldet "Drone" 
	\item[] \code{git clone https://github.com/IHA-Quadro/Drone.git}
	\end{itemize}
\item Download Githubs eget program\footnote{\url{https://help.github.com/articles/set-up-git}} (tryk på den grønne knap).
\end{enumerate}

Hvis man ikke vil bruge Github, kan projektet downloades ved tryk på sidens "Download ZIP" i højre hjørne, hvorefter man selv skal holde styr på det.

Projektfilerne kan placeres hvor man vil på computeren, så længe de beholdes i samme mappe.
Projektet der skal bruges ligger under \code{Kode/Drone/Drone.sln}

Når projektet er åbent vælges fra menuen 
\begin{itemize}
\item Tools $\rightarrow$ Visual Micro $\rightarrow$ Boards $\rightarrow$ Arduino Mega 2560 or Mega ADK
\item Tools $\rightarrow$ Visual Micro $\rightarrow$ Programmer $\rightarrow$ AVRISP mkII
\item Tools $\rightarrow$ Visual Micro $\rightarrow$ Serial Port $\rightarrow$ COM? (der er et spørgsmåltegn, for nummeret afhænger af, hvilken port USB-kablet er isat).
\end{itemize}

Herefter er projektet klar til at blive kompileret og kørt.
Vælg menuen Debug $\rightarrow$ Start Debugging.




\subsubsection{Sender og modtager}
For at programmere sender og modtager bruges Atmel Studio 6.1\footnote{\url{http://www.atmel.com/tools/atmelstudio.aspx}} \fxnote{RBK: Beskriv hvad der skal gøres for, at din del spiller}.



























\end{document}