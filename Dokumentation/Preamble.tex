
%Sidehoved og -fod pakke
%Margin
\usepackage[left=3cm,right=2cm,top=2cm,bottom=2cm]{geometry}
\usepackage{lastpage}
\usepackage{fancyhdr}					% Header-style
\pagestyle{fancy}
\fancyhf{} % slet alt
\fancyfoot[C]{Side \thepage \text{ af} \pageref{LastPage}} % sidetallet yderst
\lhead[]{\leftmark} % lige side, kapitel titel
\headheight = 35pt
\textheight = 680pt
\footskip 	= 40pt
%\rhead{
%\begin{picture}(0,0)
%\put(-120,0){ \includegraphics[scale=0.2]{../07_Administrativt/Vejledninger/au-ingenioerhoejskolen_da.jpg}}
%\end{picture}}
\renewcommand{\headrulewidth}{0.4pt}
\renewcommand{\footrulewidth}{0.4pt}


%Mellemrum mellem linjerne
\usepackage{setspace}
\setstretch{1.5} %Afstanden


%Seperated files
%--------------------------------------------------
%Opret filer således:
%\documentclass[Navn-på-hovedfil]{subfiles}
%\begin{document}
% Indmad
%\end{document}
%
% I hovedfil inkluderes således:
% \subfile{navn-på-subfil}
%--------------------------------------------------
\usepackage{subfiles}

%Standard sti at søge efter billeder
%--------------------------------------------------
%\begin{figure}[hbtp]
%\centering
%\includegraphics[scale=1]{filnavn-for-png}
%\caption{Titel}
%\label{fig:referenceNavn}
%\end{figure}
%--------------------------------------------------
\usepackage{graphicx}
\usepackage{float}
\graphicspath{{Figurer/}}

%Speciel skrift for enkelt linje kode
%--------------------------------------------------
%Udskriver med fonten 'Courier'
%Mere info her: http://tex.stackexchange.com/questions/25249/how-do-i-use-a-particular-font-for-a-small-section-of-text-in-my-document
%Eksempel: Funktionen \code{void Hello()} giver et output
%--------------------------------------------------
\newcommand{\code}[1]{{\fontfamily{pcr}\selectfont #1}}

%FixMe pakken viser små kommentarer, hvor der skal rettes
%--------------------------------------------------
%Brug med følgende: \fxnote{det her skal uddybes!}
%Se liste over alle fixMe's: \listoffixmes
%--------------------------------------------------
\usepackage[footnote,draft,danish,silent,nomargin]{fixme}


% Følgende er til koder.
%----------------------------------------------------------
%\begin{lstlisting}[caption=Overskrift på boks, style=Code-C++, label=lst:referenceLabel]
%public void hello(){}
%\end{lstlisting}
%----------------------------------------------------------

%Exstra space
\usepackage{xspace}
%Navn på bokse efterfulgt af \xspace (hvis det skal være mellemrum
%gives det med denne udvidelse. Ellers ingen mellemrum.
\newcommand{\codeTitle}{Kodeudsnit\xspace}

%Pakker der skal bruges til lstlisting
\usepackage{listings}
\usepackage{color}
\usepackage{textcomp}
\definecolor{listinggray}{gray}{0.9}
\definecolor{lbcolor}{rgb}{0.9,0.9,0.9}
\renewcommand{\lstlistingname}{\codeTitle}
\lstdefinestyle{Code}
{
	keywordstyle	= \bfseries\ttfamily\color[rgb]{0,0,1},
	identifierstyle	= \ttfamily,
	commentstyle	= \color[rgb]{0.133,0.545,0.133},
	stringstyle		= \ttfamily\color[rgb]{0.627,0.126,0.941},
	showstringspaces= false,
	basicstyle		= \small,
	numberstyle		= \footnotesize,
%	numbers			= left, % Tal? Udkommenter hvis ikke
	stepnumber		= 2,
	numbersep		= 6pt,
	tabsize			= 2,
	breaklines		= true,
	prebreak 		= \raisebox{0ex}[0ex][0ex]{\ensuremath{\hookleftarrow}},
	breakatwhitespace= false,
%	aboveskip		= {1.5\baselineskip},
  	columns			= fixed,
  	upquote			= true,
  	extendedchars	= true,
 	backgroundcolor = \color{lbcolor},
	lineskip		= 1pt,
%	xleftmargin		= 17pt,
%	framexleftmargin= 17pt,
	framexrightmargin	= 0pt, %6pt
%	framexbottommargin	= 4pt,
}

%Bredde der bruges til indryk
%Den skal være 6 pt mindre
\usepackage{calc}
\newlength{\mywidth}
\setlength{\mywidth}{\textwidth-6pt}


% Forskellige styles for forskellige kodetyper
\usepackage{caption}
\DeclareCaptionFont{white}{\color{white}}
\DeclareCaptionFormat{listing}%
{\colorbox[cmyk]{0.43, 0.35, 0.35,0.35}{\parbox{\mywidth}{\hspace{5pt}#1#2#3}}}
\captionsetup[lstlisting]
{
	format			= listing,
	labelfont		= white,
	textfont		= white, 
	singlelinecheck	= false, 
	width			= \mywidth,
	margin			= 0pt, 
	font			= {bf,footnotesize}
}

\lstdefinestyle{Code-C} {language=C, style=Code}
\lstdefinestyle{Code-Java} {language=Java, style=Code}
\lstdefinestyle{Code-C++} {language=[Visual]C++, style=Code}
\lstdefinestyle{Code-VHDL} {language=VHDL, style=Code}
\lstdefinestyle{Code-Bash} {language=Bash, style=Code}

%Text typesetting
%--------------------------------------------------------
\usepackage{baskervald}
\usepackage[T1]{fontenc}              
\usepackage[utf8]{inputenc}         
\usepackage[danish]{babel}        
\usepackage[bf,small,raggedright,compact]{titlesec}
%\usepackage[parfill]{parskip}
%\usepackage{setspace}

%Dybde på indholdsfortegnelse
%----------------------------------------------------------
%Chapter, section, subsection, subsubsection
%----------------------------------------------------------
\setcounter{secnumdepth}{4}

%Tables
%----------------------------------------------------------
\usepackage{tabularx}
\usepackage{multirow} 
\usepackage{multicol} 
\usepackage{booktabs}

%Misc
%----------------------------------------------------------
\usepackage{cite}
\usepackage{appendix}
\usepackage{amssymb}
\usepackage{url}
\usepackage{enumerate} 
\usepackage{amsmath} %Math bibliotek


\usepackage{longtable}
\usepackage{enumitem} %Styring af itemize og enumerate


% Links
\usepackage{hyperref}
\hypersetup{ 
	colorlinks	= true, 	% false: boxed links; true: colored links
    urlcolor	= blue,		% color of external links
    linkcolor	= black, 	% color of page numbers
}

%UC environment
%----------------------------------------------------------
%\newenvironment{UC}[1]
%{
%\begin{tabular}{| p{1.2cm} | p{6cm} | p{5.5cm} | l|}\hline
%%\textbf{TRIN} & \textbf{Aktion / Input} & \textbf{Forventet resultat} & \textbf{CHK} \\
%}
%{&\\\hline
%\end{tabular}
%\\\\}
%\newcommand{\usecase}[2]{{Use Case #1 & #2 \\}} 
%\newcommand\mal{&}