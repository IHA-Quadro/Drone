\documentclass[Main]{subfiles}
\begin{document}

\chapter{Indledning}

\section{Formål}

\section{Valg af trådløs kommunikation}

Der er flere krav til den trådløse sender;

\begin{itemize}
\item Regler
\item Frekvens
\item Data bus
\item Hastighed
\end{itemize}

\subsection{Regler og valg af Frekvens}
Der er mange regler inden for trådløs kommunikation, så hvilke frekvenser er lovlige at udnytte uden licens?

Det er de fleste i ISM-båndet, da de er lavet til  Industriel, videnskabelig og medicinsk bånd. Grunden til det ikke er alle er at der er lokale krav, alt efter hvor man er i verden. Eks har USA meget komplekse krav til 433MHZ.\cite[s. 32]{Lov1}

De meste gængse frekvenser der er kan benyttes i industrien og til hobby brug er derfor:

\begin{itemize}
\item 433 MHz
\item 868 MHz
\item 915 MHz
\item 2.4 GHz
\item 5.8 GHz
\end{itemize}

Så der skal vælges en frekvens af disse.
Den frekvens der egner sig bedst dette projekt, er den frekvens der rækker længst. 

\begin{figure}[H]
\centering
\includegraphics[scale=1]{FrekvensDempning}
\caption{Frekvens dæmpning - Figuren viser tests udfør af Telit wireless solutions et stort italiensk firma med speciale i trådløs teknologi.}
\end{figure}




Som man det ses på figures over så er der mindre modstand i objekter, desto længere man kommer ned i frekvens.
I og med at næsten alle hobby entusiaster inden for RC fly, bruger 2.4 GHz. 
Bliver valget for dette projekt 433 MHz.

\subsection{Valg af sender}

Når det var valgt at projektet skulle benytte 433MHz. Skulle der findes en sender/modtager.
Valget faldt på en chip fra Texas Instruments, Grundet des mange muligheder.
Nogle af de ting som projektet kunne drage fordel af var;

\begin{itemize}
\item Flere kanaler
\item Variable datahastighed (0.6 to 600 kbps)
\item Gode pakkehåndterings muligheder 
\item Flere frekvenser, hvor 433MHz er en af dem
\item Programmerbar power output
\item Spi interface
\end{itemize}

\end{document}