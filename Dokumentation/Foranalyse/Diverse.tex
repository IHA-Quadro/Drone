\documentclass[Main]{subfiles}
\begin{document}



\chapter{Diverse}
 
\section[Sikkerhed]{P9 -- Sikkerhed}
Af sikkerhedsmæssige årsager blev dronens hurtigtroterende propeller tager meget seriøst og gruppens medlemmer var enige om, at den skulle fastgøres, indtil den var sikker at flyve med.
\begin{itemize}
\item Frit svævende
	\begin{itemize}
	\item Dronen blev spændt fast i loftet med snor, så den hang i 1 meters højde. Derudover blev den spændt fast i en tung kasse på jorden, så den havde 10 cm løs snor. 
	\item Idéen var, at dronen kun skulle kunne flyve op og ned uden at flyve væk eller beskadige sine omgivelser.
	\item Desværre fløj dronen ukontrolleret i fri luft opstart og skulle derfor have lov at starte på jorden.
	\end{itemize}
	
\item Plade med kort snor
	\begin{itemize}
	\item Der blev udarbejdet en plade med en karbinhage i midten, som dronen blev fastmonteret med vha. en snor ved midten af dronens bund.
	Snoren er så kort, at når dronen tilter vil propellerne ikke kunne nå jorden og blive beskadiget.
	\item Desværre flyver dronen først stabilt i ca. 30 cm højde og der skulle derfor give mere line.
	\end{itemize}
	
\item Plade med middel snor og lang snor i loftet
	\begin{itemize}
	\item Samme plade, blot med længere snor og snor i loftet der lader dronen stå på jorden.
	\item Idéen var, at den lange snor kunne spændes til dronen sad i spænd, såfremt den begyndte at flyve mod en mur eller blot ukontrolleret.
	\item Virker godt, men er svært at teste endelig flyvning ved.
	\end{itemize}

\item Endelige test
	\begin{itemize}
	\item Dronen må ikke flyve udenfor, tæt på bebyggelse\cite{Lov1}, men siger intet om flyvning indenfor. 
	Et rum i bygningens kælder var derfor et perfekt sted at teste uden snor monteret.
	\item Lokalet havde betongulv, -væg og -loft og kun en enkelt lampe kunne beskadiges. 
	\end{itemize}
\end{itemize}

\end{document}